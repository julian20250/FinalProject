%%%%%%%%%%%%%%%%%%%%%%%%%%%%%%%%%%%%%%%%%
% Journal Article
% LaTeX Template
% Version 1.4 (15/5/16)
%
% This template has been downloaded from:
% http://www.LaTeXTemplates.com
%
% Original author:
% Frits Wenneker (http://www.howtotex.com) with extensive modifications by
% Vel (vel@LaTeXTemplates.com)
%
% License:
% CC BY-NC-SA 3.0 (http://creativecommons.org/licenses/by-nc-sa/3.0/)
%
%%%%%%%%%%%%%%%%%%%%%%%%%%%%%%%%%%%%%%%%%

%----------------------------------------------------------------------------------------
%	PACKAGES AND OTHER DOCUMENT CONFIGURATIONS
%----------------------------------------------------------------------------------------

\documentclass[twoside,twocolumn]{article}

\usepackage{blindtext} % Package to generate dummy text throughout this template 

\usepackage[sc]{mathpazo} % Use the Palatino font
\usepackage[T1]{fontenc} % Use 8-bit encoding that has 256 glyphs
\linespread{1.05} % Line spacing - Palatino needs more space between lines
\usepackage{microtype} % Slightly tweak font spacing for aesthetics
\usepackage[pdftex]{graphicx}    
\usepackage{units}
\usepackage{amsmath}
\usepackage{breqn}
\usepackage{listings}    
\usepackage[spanish]{babel}
\usepackage[utf8]{inputenc}

\usepackage[hmarginratio=1:1,top=32mm,columnsep=20pt]{geometry} % Document margins
\usepackage[hang, small,labelfont=bf,up,textfont=it,up]{caption} % Custom captions under/above floats in tables or figures
\usepackage{booktabs} % Horizontal rules in tables

\usepackage{lettrine} % The lettrine is the first enlarged letter at the beginning of the text

\usepackage{enumitem} % Customized lists
\setlist[itemize]{noitemsep} % Make itemize lists more compact
\AtBeginDocument{%
  \renewcommand\tablename{Tabla}
}
\usepackage{abstract} % Allows abstract customization
\renewcommand{\abstractnamefont}{\normalfont\bfseries} % Set the "Abstract" text to bold
\renewcommand{\abstracttextfont}{\normalfont\small\itshape} % Set the abstract itself to small italic text
\setlength\parindent{0pt}
\usepackage{titlesec} % Allows customization of titles
\renewcommand\thesection{\Roman{section}} % Roman numerals for the sections
\renewcommand\thesubsection{\roman{subsection}} % roman numerals for subsections

\titleformat{\section}[block]{\large\scshape\centering}{\thesection.}{1em}{} % Change the look of the section titles
\titleformat{\subsection}[block]{\large}{\thesubsection.}{1em}{} % Change the look of the section titles
\usepackage{xcolor}
\definecolor{light-gray}{gray}{0.95}
\newcommand{\code}[1]{\colorbox{light-gray}{\texttt{#1}}}
\usepackage{fancyhdr} % Headers and footers
\pagestyle{fancy} % All pages have headers and footers
\fancyhead{} % Blank out the default header
\fancyfoot{} % Blank out the default footer
\fancyhead[C]{Propagación de una Epidemia mediante Autómatas Celulares$\bullet$ Noviembre 2016 $\bullet$ Herramientas Computacionales} % Custom header text
\fancyfoot[RO,LE]{\thepage} % Custom footer text

\usepackage{titling} % Customizing the title section

\usepackage{hyperref} % For hyperlinks in the PDF

%----------------------------------------------------------------------------------------
%	TITLE SECTION
%----------------------------------------------------------------------------------------

\setlength{\droptitle}{-4\baselineskip} % Move the title up

\pretitle{\begin{center}\Huge\bfseries} % Article title formatting
\posttitle{\end{center}} % Article title closing formatting
\title{Propagación de una Epidemia mediante Autómatas Celulares} % Article title
\author{%
\textsc{Julián Jiménez Cárdenas} \\[1ex] % Your name
\normalsize \href{mailto:juojimenezca@unal.edu.co}{juojimenezca@unal.edu.co} % Your email address
\and % Uncomment if 2 authors are required, duplicate these 4 lines if more
\textsc{Juan Sebastián Ordóñez Soto} \\[1ex]
\normalsize \href{mailto:jsordonezs@unal.edu.co}{jsordonezs@unal.edu.co} 
% Your email address
\\ \\ \normalsize{Herramientas Computacionales, Departamento de Física, Universidad Nacional, Bogotá.} % Your institution
}

\date{\normalsize{Noviembre 29, 2016}} % Leave empty to omit a date
\renewcommand{\maketitlehookd}{%
\begin{abstract}
\normalsize{En el presente artículo se muestran los resultados obtenidos al reproducir el algoritmo descrito en \cite{b1}, en éste los autores buscaban simular la evolución de una epidemia mediante difusión utilizando un objeto llamado \emph{Autómatas Celulares}, con acciones bien definidas (reglas de evolución) en cierto intervalo de tiempo. Asimismo, se comprueba la validez del método al compararlo con el modelo \textbf{SIR}, comúnmente utilizado en el modelamiento de este fenómeno.}
\end{abstract}
}


%----------------------------------------------------------------------------------------

\begin{document}
\maketitle
\section{Introducción al Problema}
El objetivo de la simulación es reproducir el comportamiento temporal de una epidemia por difusión mediante autómatas celulares (sistema, semeja a los seres vivos), por ende, éstos deben estar ubicados en un mismo punto al principio de la simulación, para después irse moviendo a lo largo del espacio previamente delimitado y discretizado.\\ \\
Referente al autómata celular, es un objeto (en este caso, una estructura de \code{C++}) regido por reglas de evolución simples descritas en la subsección \textit{Algoritmo}. Las variables determinantes para simular el comportamiento son: el número de autómatas celulares $N$, la probabilidad que tiene un autómata infectado de curarse $PInm$, el tamaño de la malla bidimensional en la que estarán los autómatas, el número inicial de infectados, la probabilidad de que un autómata se infecte por la presencia de autómatas infectados $r$ y el tiempo de evolución del sistema (en unidades arbitrarias).\\ \\
Después de reproducir los resultados del modelo \textbf{SIR}, se analizaron algunos casos de prueba, así como la relación existente entre la probabilidad de volverse inmune siendo infectado $PInm$ contra el número final de autómatas susceptibles.\\ \\
\subsection{Algoritmo}
Los autómatas celulares fueron programados para seguir las siguientes reglas. En cada paso de tiempo se deben realizar las siguientes acciones.
\subsubsection{Movimiento Difusivo}
\begin{enumerate}
\item La célula rotará con igual probabilidad hacia arriba, abajo, derecha o izquierda.
\item La célula se moverá en la dirección a la cual apunta. Hay que tener precaución, pues la célula no se debe salir de los límites del mapa.
\end{enumerate}
\subsection{Contagio Epidémico}
\begin{enumerate}
\item Si la célula está inmune sigue inmune.
\item Si está infectada se cura con probabilidad $PInm$.
\item Si está sano y tiene infectados en el mismo punto o a sus alrededores, se calcula la probabilidad que éste tiene de infectarse proporcionalmente al número de infectados alrededor suyo. Mediante dicha probabilidad se determinará si se infecta la célula o no.
\end{enumerate}
Cabe hacer énfasis en que las celdas en las que se ubican los autómatas celulares y éstos son distintos.

\section{Código}
Los programas encargados de calcular el comportamiento del autómatas celulares se hicieron en \code{C++}. De éstos se generaban archivos de texto plano, para ser posteriormente leídos y relacionados mediante scripts de \code{Python}.
\begin{thebibliography}{99} % Bibliography - this is intentionally simple in this template

%\bibitem[Figueredo and Wolf, 2009]{Figueredo:2009dg}
%Figueredo, A.~J. and Wolf, P. S.~A. (2009).
%\newblock Assortative pairing and life history strategy - a cross-cultural
%  study.
%\newblock {\em Human Nature}, 20:317--330.
\bibitem[1]{b1}W. F. Oquendo y J. D. Muñoz.  \emph{Simulación de la Propagación de una Epidemia Utilizando un Autómata Celular de Difusión Bidimensional}; Revista Colombiana de Física, Vol40, No.2, Julio 2008.

\end{thebibliography}
\end{document}